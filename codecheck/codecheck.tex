\documentclass[fontsize=10pt,parskip]{scrartcl}


    

    % Basic figure setup, for now with no caption control since it's done
    % automatically by Pandoc (which extracts ![](path) syntax from Markdown).
    \usepackage{graphicx}
    % Keep aspect ratio if custom image width or height is specified
    \setkeys{Gin}{keepaspectratio}
    % Maintain compatibility with old templates. Remove in nbconvert 6.0
    \let\Oldincludegraphics\includegraphics
    % Ensure that by default, figures have no caption (until we provide a
    % proper Figure object with a Caption API and a way to capture that
    % in the conversion process - todo).
    \usepackage{caption}
    \DeclareCaptionFormat{nocaption}{}
    \captionsetup{format=nocaption,aboveskip=0pt,belowskip=0pt}

    \usepackage{float}
    \floatplacement{figure}{H} % forces figures to be placed at the correct location
    \usepackage{xcolor} % Allow colors to be defined
    \usepackage{enumerate} % Needed for markdown enumerations to work
    \usepackage{geometry} % Used to adjust the document margins
    \usepackage{amsmath} % Equations
    \usepackage{amssymb} % Equations
    \usepackage{textcomp} % defines textquotesingle
    % Hack from http://tex.stackexchange.com/a/47451/13684:
    \AtBeginDocument{%
        \def\PYZsq{\textquotesingle}% Upright quotes in Pygmentized code
    }
    \usepackage{upquote} % Upright quotes for verbatim code
    \usepackage{eurosym} % defines \euro

    \usepackage{iftex}
    \ifPDFTeX
        \usepackage[T1]{fontenc}
        \IfFileExists{alphabeta.sty}{
              \usepackage{alphabeta}
          }{
              \usepackage[mathletters]{ucs}
              \usepackage[utf8x]{inputenc}
          }
    \else
        \usepackage{fontspec}
        \usepackage{unicode-math}
    \fi

    \usepackage{fancyvrb} % verbatim replacement that allows latex
    \usepackage{grffile} % extends the file name processing of package graphics
                         % to support a larger range
    \makeatletter % fix for old versions of grffile with XeLaTeX
    \@ifpackagelater{grffile}{2019/11/01}
    {
      % Do nothing on new versions
    }
    {
      \def\Gread@@xetex#1{%
        \IfFileExists{"\Gin@base".bb}%
        {\Gread@eps{\Gin@base.bb}}%
        {\Gread@@xetex@aux#1}%
      }
    }
    \makeatother
    \usepackage[Export]{adjustbox} % Used to constrain images to a maximum size
    \adjustboxset{max size={0.9\linewidth}{0.9\paperheight}}

    % The hyperref package gives us a pdf with properly built
    % internal navigation ('pdf bookmarks' for the table of contents,
    % internal cross-reference links, web links for URLs, etc.)
    \usepackage{hyperref}
    % The default LaTeX title has an obnoxious amount of whitespace. By default,
    % titling removes some of it. It also provides customization options.
    \usepackage{titling}
    \usepackage{longtable} % longtable support required by pandoc >1.10
    \usepackage{booktabs}  % table support for pandoc > 1.12.2
    \usepackage{array}     % table support for pandoc >= 2.11.3
    \usepackage{calc}      % table minipage width calculation for pandoc >= 2.11.1
    \usepackage[inline]{enumitem} % IRkernel/repr support (it uses the enumerate* environment)
    \usepackage[normalem]{ulem} % ulem is needed to support strikethroughs (\sout)
                                % normalem makes italics be italics, not underlines
    \usepackage{soul}      % strikethrough (\st) support for pandoc >= 3.0.0
    \usepackage{mathrsfs}
    

    

    % Colors for the hyperref package
    \definecolor{urlcolor}{rgb}{0,.145,.698}
    \definecolor{linkcolor}{rgb}{.71,0.21,0.01}
    \definecolor{citecolor}{rgb}{.12,.54,.11}

    % ANSI colors
    \definecolor{ansi-black}{HTML}{3E424D}
    \definecolor{ansi-black-intense}{HTML}{282C36}
    \definecolor{ansi-red}{HTML}{E75C58}
    \definecolor{ansi-red-intense}{HTML}{B22B31}
    \definecolor{ansi-green}{HTML}{00A250}
    \definecolor{ansi-green-intense}{HTML}{007427}
    \definecolor{ansi-yellow}{HTML}{DDB62B}
    \definecolor{ansi-yellow-intense}{HTML}{B27D12}
    \definecolor{ansi-blue}{HTML}{208FFB}
    \definecolor{ansi-blue-intense}{HTML}{0065CA}
    \definecolor{ansi-magenta}{HTML}{D160C4}
    \definecolor{ansi-magenta-intense}{HTML}{A03196}
    \definecolor{ansi-cyan}{HTML}{60C6C8}
    \definecolor{ansi-cyan-intense}{HTML}{258F8F}
    \definecolor{ansi-white}{HTML}{C5C1B4}
    \definecolor{ansi-white-intense}{HTML}{A1A6B2}
    \definecolor{ansi-default-inverse-fg}{HTML}{FFFFFF}
    \definecolor{ansi-default-inverse-bg}{HTML}{000000}

    % common color for the border for error outputs.
    \definecolor{outerrorbackground}{HTML}{FFDFDF}

    % commands and environments needed by pandoc snippets
    % extracted from the output of `pandoc -s`
    \providecommand{\tightlist}{%
      \setlength{\itemsep}{0pt}\setlength{\parskip}{0pt}}
    \DefineVerbatimEnvironment{Highlighting}{Verbatim}{commandchars=\\\{\}}
    % Add ',fontsize=\small' for more characters per line
    \newenvironment{Shaded}{}{}
    \newcommand{\KeywordTok}[1]{\textcolor[rgb]{0.00,0.44,0.13}{\textbf{{#1}}}}
    \newcommand{\DataTypeTok}[1]{\textcolor[rgb]{0.56,0.13,0.00}{{#1}}}
    \newcommand{\DecValTok}[1]{\textcolor[rgb]{0.25,0.63,0.44}{{#1}}}
    \newcommand{\BaseNTok}[1]{\textcolor[rgb]{0.25,0.63,0.44}{{#1}}}
    \newcommand{\FloatTok}[1]{\textcolor[rgb]{0.25,0.63,0.44}{{#1}}}
    \newcommand{\CharTok}[1]{\textcolor[rgb]{0.25,0.44,0.63}{{#1}}}
    \newcommand{\StringTok}[1]{\textcolor[rgb]{0.25,0.44,0.63}{{#1}}}
    \newcommand{\CommentTok}[1]{\textcolor[rgb]{0.38,0.63,0.69}{\textit{{#1}}}}
    \newcommand{\OtherTok}[1]{\textcolor[rgb]{0.00,0.44,0.13}{{#1}}}
    \newcommand{\AlertTok}[1]{\textcolor[rgb]{1.00,0.00,0.00}{\textbf{{#1}}}}
    \newcommand{\FunctionTok}[1]{\textcolor[rgb]{0.02,0.16,0.49}{{#1}}}
    \newcommand{\RegionMarkerTok}[1]{{#1}}
    \newcommand{\ErrorTok}[1]{\textcolor[rgb]{1.00,0.00,0.00}{\textbf{{#1}}}}
    \newcommand{\NormalTok}[1]{{#1}}

    % Additional commands for more recent versions of Pandoc
    \newcommand{\ConstantTok}[1]{\textcolor[rgb]{0.53,0.00,0.00}{{#1}}}
    \newcommand{\SpecialCharTok}[1]{\textcolor[rgb]{0.25,0.44,0.63}{{#1}}}
    \newcommand{\VerbatimStringTok}[1]{\textcolor[rgb]{0.25,0.44,0.63}{{#1}}}
    \newcommand{\SpecialStringTok}[1]{\textcolor[rgb]{0.73,0.40,0.53}{{#1}}}
    \newcommand{\ImportTok}[1]{{#1}}
    \newcommand{\DocumentationTok}[1]{\textcolor[rgb]{0.73,0.13,0.13}{\textit{{#1}}}}
    \newcommand{\AnnotationTok}[1]{\textcolor[rgb]{0.38,0.63,0.69}{\textbf{\textit{{#1}}}}}
    \newcommand{\CommentVarTok}[1]{\textcolor[rgb]{0.38,0.63,0.69}{\textbf{\textit{{#1}}}}}
    \newcommand{\VariableTok}[1]{\textcolor[rgb]{0.10,0.09,0.49}{{#1}}}
    \newcommand{\ControlFlowTok}[1]{\textcolor[rgb]{0.00,0.44,0.13}{\textbf{{#1}}}}
    \newcommand{\OperatorTok}[1]{\textcolor[rgb]{0.40,0.40,0.40}{{#1}}}
    \newcommand{\BuiltInTok}[1]{{#1}}
    \newcommand{\ExtensionTok}[1]{{#1}}
    \newcommand{\PreprocessorTok}[1]{\textcolor[rgb]{0.74,0.48,0.00}{{#1}}}
    \newcommand{\AttributeTok}[1]{\textcolor[rgb]{0.49,0.56,0.16}{{#1}}}
    \newcommand{\InformationTok}[1]{\textcolor[rgb]{0.38,0.63,0.69}{\textbf{\textit{{#1}}}}}
    \newcommand{\WarningTok}[1]{\textcolor[rgb]{0.38,0.63,0.69}{\textbf{\textit{{#1}}}}}
    \makeatletter
    \newsavebox\pandoc@box
    \newcommand*\pandocbounded[1]{%
      \sbox\pandoc@box{#1}%
      % scaling factors for width and height
      \Gscale@div\@tempa\textheight{\dimexpr\ht\pandoc@box+\dp\pandoc@box\relax}%
      \Gscale@div\@tempb\linewidth{\wd\pandoc@box}%
      % select the smaller of both
      \ifdim\@tempb\p@<\@tempa\p@
        \let\@tempa\@tempb
      \fi
      % scaling accordingly (\@tempa < 1)
      \ifdim\@tempa\p@<\p@
        \scalebox{\@tempa}{\usebox\pandoc@box}%
      % scaling not needed, use as it is
      \else
        \usebox{\pandoc@box}%
      \fi
    }
    \makeatother

    % Define a nice break command that doesn't care if a line doesn't already
    % exist.
    \def\br{\hspace*{\fill} \\* }
    % Math Jax compatibility definitions
    \def\gt{>}
    \def\lt{<}
    \let\Oldtex\TeX
    \let\Oldlatex\LaTeX
    \renewcommand{\TeX}{\textrm{\Oldtex}}
    \renewcommand{\LaTeX}{\textrm{\Oldlatex}}
    % Document parameters
    % Document title
    \title{codecheck}
    
    
    
    
    
    
    
\usepackage{fvextra}
\DefineVerbatimEnvironment{Verbatim}{Verbatim}{fontsize=\footnotesize,breaklines=true}


    
    % Prevent overflowing lines due to hard-to-break entities
    \sloppy
    % Setup hyperref package
    \hypersetup{
      breaklinks=true,  % so long urls are correctly broken across lines
      colorlinks=true,
      urlcolor=urlcolor,
      linkcolor=linkcolor,
      citecolor=citecolor,
      }
    % Slightly bigger margins than the latex defaults
    

    

\begin{document}
    
    
    
    

    
    Template document for \href{https://codecheck.org.uk/}{CODECHECK}'s with
a Jupyter notebook. Note that the files in this repository should be
placed into a \texttt{codecheck} directory, and require a
\texttt{codecheck.yml} file in the main folder
(i.e.~\texttt{../codecheck.yml} from the point of view of this
notebook). See the
\href{https://codecheck.org.uk/guide/community-process}{CODECHECK
community process guide} for more details.

Note that the \texttt{\{-\}} at the end of the headings is necessary to
avoid numbered headings in jupyter's \texttt{nbexport}.

    \subsection*{Validation (Optional)}\label{validation-optional}
\addcontentsline{toc}{subsection}{Validation (Optional)}

The following cells run validation checks on the \texttt{codecheck.yml}
file. This is optional but recommended to catch configuration errors
early. You can skip these cells if you prefer.

    \begin{Verbatim}[commandchars=\\\{\}]
Validation failed with 1 issue(s)

    \end{Verbatim}

    
    \subsection{x Errors (1)}\label{x-errors-1}

\begin{itemize}
\tightlist
\item
  \textbf{codechecker}: Codechecker must be a dictionary, got list

  \begin{itemize}
  \tightlist
  \item
    \emph{Suggestion}: Structure codechecker as: \{name: `\ldots{}',
    ORCID: `\ldots{}'\}
  \end{itemize}
\end{itemize}

    

    
    \section*{CODECHECK certificate
2025-026}\label{codecheck-certificate-2025-026}
\addcontentsline{toc}{section}{CODECHECK certificate 2025-026}

\subsection*{\texorpdfstring{\href{https://doi.org/10.5281/zenodo.4720843}{doi.org/10.5281/zenodo.4720843}}{doi.org/10.5281/zenodo.4720843}}\label{doi.org10.5281zenodo.4720843}
\addcontentsline{toc}{subsection}{{doi.org/10.5281/zenodo.4720843}}

\href{https://codecheck.org.uk}{\pandocbounded{\includegraphics[keepaspectratio,alt={CODECHECK logo}]{codecheck_logo.png}}}

    

    \begin{center}\rule{0.5\linewidth}{0.5pt}\end{center}

    \subsection*{CODECHECK summary}\label{codecheck-summary}
\addcontentsline{toc}{subsection}{CODECHECK summary}

    
    {\def\LTcaptype{none} % do not increment counter
\begin{longtable}[]{@{}
  >{\raggedright\arraybackslash}p{(\linewidth - 2\tabcolsep) * \real{0.4444}}
  >{\raggedright\arraybackslash}p{(\linewidth - 2\tabcolsep) * \real{0.5556}}@{}}
\toprule\noalign{}
\begin{minipage}[b]{\linewidth}\raggedright
Item
\end{minipage} & \begin{minipage}[b]{\linewidth}\raggedright
Value
\end{minipage} \\
\midrule\noalign{}
\endhead
\bottomrule\noalign{}
\endlastfoot
Title & \emph{Automated validation of route instructions in indoor
environments} \\
Authors & Reza Arabsheibani (ORCID:
\href{https://orcid.org/0000-0003-2224-3823}{0000-0003-2224-3823}),
Stephan Winter (ORCID:
\href{https://orcid.org/0000-0002-3403-6939}{0000-0002-3403-6939}),
Martin Tomko (ORCID:
\href{https://orcid.org/0000-0002-5736-4679}{0000-0002-5736-4679}) \\
Reference &
\href{https://doi.org/10.5311/JOSIS.2025.30.385}{doi.org/10.5311/JOSIS.2025.30.385} \\
Repository &
\href{https://github.com/codecheckers/certificate-2025-026}{github.com/codecheckers/certificate-2025-026} \\
Codechecker & Linus Dexter Hackel (ORCID:
\href{https://orcid.org/0009-0000-0114-8005}{0009-0000-0114-8005}) \\
Date of check & 2025-12-11 \\
Summary & Until now only the basic codecheck.yml has been created. \\
\end{longtable}
}

    

    \subsection*{Summary of output files
generated}\label{summary-of-output-files-generated}
\addcontentsline{toc}{subsection}{Summary of output files generated}

    
    {\def\LTcaptype{none} % do not increment counter
\begin{longtable}[]{@{}
  >{\raggedright\arraybackslash}p{(\linewidth - 4\tabcolsep) * \real{0.3333}}
  >{\raggedright\arraybackslash}p{(\linewidth - 4\tabcolsep) * \real{0.5455}}
  >{\raggedleft\arraybackslash}p{(\linewidth - 4\tabcolsep) * \real{0.1212}}@{}}
\toprule\noalign{}
\begin{minipage}[b]{\linewidth}\raggedright
File~~~
\end{minipage} & \begin{minipage}[b]{\linewidth}\raggedright
Comment~~~~\&nbsp
\end{minipage} & \begin{minipage}[b]{\linewidth}\raggedleft
Size (b)
\end{minipage} \\
\midrule\noalign{}
\endhead
\bottomrule\noalign{}
\endlastfoot
\texttt{AEY.jpeg} & Just a random file from the provided code repository
& 1728 \\
\end{longtable}
}

    

    \subsection*{Summary}\label{summary}
\addcontentsline{toc}{subsection}{Summary}

    
    Until now only the basic codecheck.yml has been created.

    

    \subsection*{CODECHECKER notes}\label{codechecker-notes}
\addcontentsline{toc}{subsection}{CODECHECKER notes}

\emph{TODO}

    \subsection*{Recommendations to the
authors}\label{recommendations-to-the-authors}
\addcontentsline{toc}{subsection}{Recommendations to the authors}

\emph{TODO}

    \subsection*{Citing this document}\label{citing-this-document}
\addcontentsline{toc}{subsection}{Citing this document}

    
    Linus Dexter Hackel (2025). CODECHECK Certificate 2025-026. Zenodo.
\href{https://doi.org/10.5281/zenodo.4720843}{doi.org/10.5281/zenodo.4720843}

    

    \subsection*{About CODECHECK}\label{about-codecheck}
\addcontentsline{toc}{subsection}{About CODECHECK}

    
    This certificate confirms that the codechecker could independently
reproduce the results of a computational analysis given the data and
code from a third party. A CODECHECK does not check whether the original
computation analysis is correct. However, as all materials required for
the reproduction are freely availableby following the links in this
document, the reader can then study for themselves the code and data.

    

    \subsection*{About this document}\label{about-this-document}
\addcontentsline{toc}{subsection}{About this document}

This document was created using a \href{https://jupyter.org/}{jupyter
notebook} and converted into PDF via
\href{https://nbconvert.readthedocs.io/}{nbconvert},
\href{https://pandoc.org/}{pandoc}, and
\href{http://xetex.sourceforge.net/}{xelatex}.

\subsection*{License}\label{license}
\addcontentsline{toc}{subsection}{License}

The code, data, and figures created by the original authors are licensed
under the \ldots{} license (see their
\href{https://github.com/codecheckers/causality-review/blob/main/LICENSE}{LICENSE
file}). The content of the \texttt{codecheck} directory and this report
are licensed under the \ldots{} license.

    \subsection*{Manifest files}\label{manifest-files}
\addcontentsline{toc}{subsection}{Manifest files}

\subsubsection*{CSV files}\label{csv-files}
\addcontentsline{toc}{subsubsection}{CSV files}

    

    
    \subsubsection*{Figures}\label{figures}
\addcontentsline{toc}{subsubsection}{Figures}

    

    

    % Add a bibliography block to the postdoc
    
    
    
\end{document}
